\subsection{Module Details}
Modules were designed to be as flexible as possible.
This has a few implications:

\subsubsection*{Scope contents}
Scopes mainly consist of \texttt{Func} declarations, but can be used for everything.
They are equivalent to top-level scope. (TODO: citation needed) 
We choose not to artificially limit the scope to maximize expressivity.

\subsubsection*{Modules declare new kinds}
Simplest solution was to reuse the \texttt{TypeFunc} system for compile-time tasks.
Kinds are a compile-time mechanism, and TypeFuncs are used for all compile-time tasks.
(TODO: find more arguments)

\subsubsection*{Inheritance propegation}
difference between imports and inherits.
Inherit only imports target, not the things the target inherits.
This means the namespace remains clean. (TODO: fact-check)

\subsubsection*{Modules can have premises}
'cause why not. (TODO: find simple usecase)

\subsubsection*{Module inheritance vs composition}
monad class; zero class; type monadZero ∈ monad, zero.
(TODO: good arguments for inheritance)
