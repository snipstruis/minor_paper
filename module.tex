\subsection{Module Details}
Modules were designed to be as flexible as possible, to allow them to model as great a range of abstractions as possible.
This means minimizing restrictions, and maximizing orthogonality.
This philosophy manifests itself in the following ways:

\subsubsection*{Scope contents}
Scopes mainly consist of \texttt{Func} declarations, but can be used for everything.
They are equivalent to top-level scope\footnote{Except for \texttt{import} statements, those are not allowed in modules.}. 
We choose not to artificially limit the scope to maximize expressivity.

\subsubsection*{Modules declare new kinds}
Simplest solution was to reuse the \texttt{TypeFunc} system for compile-time tasks.
Kinds are a compile-time mechanism, and TypeFuncs are used for all compile-time tasks.

\subsubsection*{Inheritance propegation}
Inherit statements inherit recursively.
If C inherits from B, and B inherits from A, then C inherits both B and A.
This is different from the way imports work.
Import statements don't import recursively.
This is to prevent the namespace getting cluttered with dependencies.

\subsubsection*{Modules can have premises}
Modules can have premises just like Funcs and TypeFuncs.
This means you can arbitrarily manipulate the input arguments for maximum expressiveness.
Again the restriction here is the inability to pass term-level arguments to modules, as that would require dependent types.

\subsubsection*{Inheritance vs composition}
Meta-Casanova uses an inheritance-based approach as opposed to a composition-based approach.

In the inheritance-based approach, kinds inherit from each other and the users of the kinds only accept one specific kind.
In the case of MonadZero, MonadZero inherits from Monad and Zero, and Funcs written for MonadZero \textit{only} accept MonadZero or derivatives.

In the composition-based approach, kinds do not inherit from each other, and the users of the kinds accept a range of kinds.
In the case of MonadZero, there would only be a Monad module and a Zero module and the Funcs that needed both would specify that in their signature.

Both strategies have their merits and are used in programming languages successfully.
The advantage of a composition-based approach is that you only have to specify the building blocks,
and don't have to specify all the combinations.
The advantage of a inheritance-based approach is that you can add special behavior to your modules.

Because flexibility is the main concern, Meta-Casanova uses inheritance.
