\section{Details}
\section{Meta-Casanova}
\begin{multicols}{2}
  Meta-casanova uses a syntax similar to natural deduction.
  A rule is comprised of a line with below it on the left of the arrow the input, and on the right the output.
  Above the line are premises that are executed sequentially.

  TODO: explain matches

  Multiple rules may match at any given time.
  If this happens, the program execution splits into multiple branches.
  Each following one rule.
  If none of the rules match, the branch dies off. 
  
  This branch-and-die paradigm is very useful for writing parsers.
  For example: when a numeric digit is encountered, it is unclear whether an integer is being parsed or when a floating point literal is being parsed. 
  In Meta-Casanova, you just write two rules. One that matches digits followed by a delimiter, and another that matches digits followed by a dot and optionally more digits.
\begin{lstlisting}[caption=A Boolean expression interpreter in Meta-Casanova.]
  Data "TRUE"  → Value
  Data "FALSE" → Value

  Func "eval" → Expr → Value

  Data Expr → "&" → Expr → Expr
  Data Expr → "|" → Expr → Expr


  eval a → TRUE
  eval b → res
  ────────────────
  eval (a&b) → res
  

  eval a → FALSE
  ──────────────────
  eval (a&b) → FALSE
 

  eval a → TRUE
  ─────────────────
  eval (a|b) → TRUE


  eval a → FALSE
  eval b → res
  ────────────────
  eval (a|b) → res
\end{lstlisting}

\end{multicols}

\pagebreak

\subsection{TypeFunc Details}
\begin{multicols}{2}\noindent
  TypeFuncs were designed to be consistent with the already existing Funcs.
  This has a few implications:

  \subsubsection*{Everything curries}
  Everything curries because it was the most consistent from a language perspective.
  This also gives the compiler a uniform way to handle application.
  
  \subsubsection*{Typefuncs can have premises}
  Maximum expressivity. (TODO: use case here)
  General and consistent system that works everywhere.

  \subsubsection*{Double arrows (⇒)}
  Double arrows to clearly differentiate run-time from compile-time.
  Unlike Haskell (they unified types and kinds)\footnote{source}

  \subsubsection*{Dependent Types}
  A future inclusion of dependent types is possible.
  It would mean that the type namespace will be merged into the module namespace.
  This won't break backward compatibility, since identifiers need to be unique.
\end{multicols}

\pagebreak
\subsection{Module Details}
\begin{multicols}{2}\noindent
  Modules were designed to be as flexible as possible.
  This has a few implications:

  \subsubsection*{Scope contents}
  Scopes mainly consist of \texttt{Func} declarations, but can also be used for other things.

  \subsubsection*{Modules declare new kinds}
  Simplest solution was to reuse the \texttt{TypeFunc} system for compile-time tasks.
  Possible problem in future with Dependent types: kind name space will then consist of module names and type names.

  \subsubsection*{Inheritance propegation}
  difference between imports and inherits.

  \subsubsection*{Modules can have premises}
  'cause why not. (TODO: find simple usecase)

  \subsubsection*{Module inheritance vs composition}
  monad class; zero class; type monadZero ∈ monad, zero.

\end{multicols}


