\section{Details}
\section{Meta-Casanova}
\begin{multicols}{2}
  Meta-casanova uses a syntax similar to natural deduction.
  A rule is comprised of a line with below it on the left of the arrow the input, and on the right the output.
  Above the line are premises that are executed sequentially.

  TODO: explain matches

  Multiple rules may match at any given time.
  If this happens, the program execution splits into multiple branches.
  Each following one rule.
  If none of the rules match, the branch dies off. 
  
  This branch-and-die paradigm is very useful for writing parsers.
  For example: when a numeric digit is encountered, it is unclear whether an integer is being parsed or when a floating point literal is being parsed. 
  In Meta-Casanova, you just write two rules. One that matches digits followed by a delimiter, and another that matches digits followed by a dot and optionally more digits.
\begin{lstlisting}[caption=A Boolean expression interpreter in Meta-Casanova.]
  Data "TRUE"  → Value
  Data "FALSE" → Value

  Func "eval" → Expr → Value

  Data Expr → "&" → Expr → Expr
  Data Expr → "|" → Expr → Expr


  eval a → TRUE
  eval b → res
  ────────────────
  eval (a&b) → res
  

  eval a → FALSE
  ──────────────────
  eval (a&b) → FALSE
 

  eval a → TRUE
  ─────────────────
  eval (a|b) → TRUE


  eval a → FALSE
  eval b → res
  ────────────────
  eval (a|b) → res
\end{lstlisting}

\end{multicols}

\pagebreak

\subsection{TypeFunc Details}
\begin{multicols}{2}\noindent
  TypeFuncs were designed to be consistent with the already existing Funcs.
  This has a few implications:

  \subsubsection*{TypeFuncs curry}
  like Funcs, TypeFuncs curry.

  Currying enables partially application.
  For example, imagine we have a state monad.
  
  \begin{code}
  TypeFunc "StateMonad" ⇒ ∗ ⇒ ∗ ⇒ ∗
  \end{code}

  \noindent Where the first argument represents the state type and the second the return type.
  This means you can define a new state monad that already has its state-type specified by writing \verb|Monad(StateMonad int)|.

  It also gives the compiler a uniform way to handle type and function application,
  so this feature increases orthogonality while not increasing compiler complexity.
  
  \subsubsection*{Dependent Types}
  Dependent types are types that depend on terms, and are currently not supported by Meta-Casanova.
  An example for this is the constant-length array, where the length of the array is embedded into the type at compile-time.
  
  A future inclusion of dependent types is possible.
  It would mean that the type namespace will be merged into the module namespace.
  This won't break backward compatibility, since all identifiers in Meta-Casanova need to be unique.

  \subsubsection*{Double arrows (⇒)}
  We choose double arrows for typefuncs to clearly differentiate them from function application(→).
  This distinction is to seperate run-time from compile-time.

  %This is unlike Haskell, where the developers on the GHC mailing list decided to unify types and kinds\footnote{source}.
  %Unlike Haskell, they unified types and kinds\footnote{source} because there was no need for a distinction there.

  \subsubsection*{Typefuncs can have premises}
  This means that Typefuncs can behave just like rules.
  This also gives us the ability to compute arbitrary expressions at compile-time.
  The only limit is that values can't be passed at compile-time, as that would require dependent types.

\end{multicols}

\pagebreak
\subsection{Module Details}
\begin{multicols}{2}\noindent
  Modules were designed to be as flexible as possible.
  This has a few implications:

  \subsubsection*{Scope contents}
  Scopes mainly consist of \texttt{Func} declarations, but can be used for everything.
  They are equivalent to top-level scope. (TODO: citation needed) 
  We choose not to artificially limit the scope to maximize expressivity.

  \subsubsection*{Modules declare new kinds}
  Simplest solution was to reuse the \texttt{TypeFunc} system for compile-time tasks.
  Kinds are a compile-time mechanism, and TypeFuncs are used for all compile-time tasks.
  (TODO: find more arguments)

  \subsubsection*{Inheritance propegation}
  difference between imports and inherits.
  Inherit only imports target, not the things the target inherits.
  This means the namespace remains clean. (TODO: fact-check)

  \subsubsection*{Modules can have premises}
  'cause why not. (TODO: find simple usecase)

  \subsubsection*{Module inheritance vs composition}
  monad class; zero class; type monadZero ∈ monad, zero.
  (TODO: good arguments for inheritance)

\end{multicols}


