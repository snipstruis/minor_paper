\section*{TypeFuncs and Module declatations}
\begin{multicols}{2}
\begin{lstlisting}[caption=A generic monad transformer]
TypeFunc "Monad" ⇒ (* ⇒ *) ⇒ Module
Monad 'M ⇒ Module {
  Func 'a → ">>=" → ('a → 'M 'b) → 'M 'b
  Func "return" → 'a → 'M 'a
}
\end{lstlisting}
\columnbreak
Maybe explain kinds.
Explain TypeFuncs.
Explain how modules are kind-level.
Typefuncs curry-able like Funcs.
\end{multicols}

\section*{module-instantiations}
\begin{lstlisting}[caption=F\#'s option monad]
  Data "Some" → 'a → OptionT 'a
  Data "None"      → OptionT 'a

  TypeFunc "Option" ⇒ * ⇒ Monad
  Option 'a ⇒ Monad {
    
  }
\end{lstlisting}

\section*{TypeAliases}
data is actually FuncAlias(two-way). Type-Alias is equivalent for types.

\section*{ArrowFuncs}
in order to represent the monad bind operator (\texttt{>>=}) as an arrow, ArrowFuncs.
