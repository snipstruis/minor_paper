\section{The Idea}
  In order to provide the typesystem we require, additional syntax is needed.
  We need a way to define type operators and a way to create \textit{kinds}.

  \subsubsection*{Kinds}
  Kinds are to types as types are to terms.\cite[Chapter~30]{Pierce02}
  They are ``one level up'' from types.
  The most used kind is \texttt{∗} (pronounced ``type'') and represents all proper types.
  Type operators are indicated with \texttt{⇒} to differentiate them from type-signatures (\texttt{→}).
  For example a few types with their respective kinds:
  \begin{code}
  int             ∗
  float           ∗
  pair            ∗ ⇒ ∗ ⇒ ∗
  pair int        ∗ ⇒ ∗ 
  pair float int  ∗
  pair pair       error
  \end{code}

  \texttt{pair pair} is not a valid type, because type \texttt{pair} has kind \texttt{∗⇒∗⇒∗}.
  It only accepts two types with kind \texttt{∗}.
  Kinds are necessary to prevent these nonsensical types from typechecking.

  Kinds --unlike types-- have no run-time representation.
  They don't need it, they are only executed at compile-time to check the validity of the types.

  This property of kinds can be used to compute arbitrary expressions at compile-time.
  In Meta-Casanova, these expressions are called \texttt{TypeFunc}s, as they operate on types.

  \subsubsection*{Modules}
  Modules in Meta-Casanova are a compile-time mechanism that allows the definition of interfaces.
  It re-uses \texttt{TypeFunc}s to declare new kinds.

  To illustrate modules, let's implement a generic monad.
  We first define the monad interface.
  
  \begin{code}
  TypeFunc "Monad" ⇒ #\verb|(∗ ⇒ ∗)|# ⇒ Module
  Monad 'M ⇒ Module {
    Func 'a →">>=" → ('a →'M 'b) →'M 'b
    Func "return" →'a →'M 'a
  }
  \end{code}

  Now that we have the interface, let's instantiate an identity monad.
  To use the interface, we first declare a type that fits the first argument.
  It has kind \verb|(∗ ⇒ ∗)| and represents the monad constructor.
  In case of the identity monad this is trivial.

  \begin{code}
  TypeFunc "IdCons" ⇒ ∗ ⇒ ∗
  Id 'a ⇒ 'a
  \end{code}

  The first line declares that we have a new type operator Id that takes a type and returns a type.
  The second line implements it. 

  To instantiate the interface we create a type operator that takes no arguments and gives us a Monad.
  
  \begin{code}
  TypeFunc "id" ⇒ Monad 
  id ⇒ Monad IdCons {
    x >>= k → k x
    return x → x
  }
  \end{code}

  We have now successfully implemented a monad that does nothing.
  While it might look useless, it is actually used quite often.
  When you pass an identity monad to a monad transformer, you get the corresponding monad.
  When you pass the identity monad to a state monad transformer, you get a state monad.

  In fact, when using monad transformers, the identity monad forms the foundation of every monad.
  At the end of the chain of monad transformers, there is always an identity monad where all those transformations are applied to.
  
  \subsubsection*{Inheritance}
  (TODO: better example)
  Modules can inherit from each other.
  This is done with the \verb|Inherit| keyword. 
  If --for instance-- we want to extend our monad module with Zero,
  we can define a new module MonadZero that inherits from Monad.

  \begin{code}
  TypeFunc "MonadZero" #\verb|⇒(∗ ⇒ ∗)⇒|#Module
  MonadZero ⇒ Monad 'M {
    Func "Zero" → 'a
  }
  \end{code}

  This way, we have easily extended Monad.
  All functions that operate on monads can operate on things derived from monads.
