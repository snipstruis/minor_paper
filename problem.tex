\section{The Problem}
  The goal of this paper is to extend Meta-Casanova's typesystem to the point where it can handle monad-transformers.
  In this section we present an extention of the type system of Meta-Casanova.
  We start by giving an overview of monadic parsers,
  then we move on to monad transformers,
  and we conclude with a look at what typesystem we need.
  
  \subsubsection*{Monadic Parsing}
  Parser monads are a useful tool for writing compilers.
  They are not only useful in the lexing stage, as they can be used for parsing intermediate representations.
  Such intermediate representations appear between different sections of the compiler and are often tree-like.
  An example would be the typechecker, which has to parse the abstract syntax tree from the front-end.

  The best aspect about monadic parsing is that it is \textit{composable}.
  It is possible to combine multiple parsers into a single parser.
  For example: there might be a parser (\texttt{digit}) that can parse a single digit,
  and a parser combinator (\texttt{repeat}) that can apply a parser multiple times until it fails.
  It is then trivial to combine these into a new parser that parses multiple digits.

  \begin{code}[language=Caml]
  let digits = repeat digit
  \end{code}

  A parser is a function that takes an input stream and a context and returns a result.
  The input stream is a sequence of characters that has yet to be parsed.
  The context is the state that is kept and modified during the parsing.
  The result is either the succesful outcome of the parser, or an error message.
  A generic implementation may look like this.

  \begin{code}[language=Caml]
  type Parser<'char,'ctxt,'result> =
    List<'char> * 'ctxt
      -> Result<'char,'ctxt,'result>
  \end{code}
  
  The Result contains --if the parser was successful-- the value returned by the parser, the modified context, and the new input stream position.
  
  \begin{code}[language=Caml]
  type Result<'char,'ctxt,'result> = 
    | Done  of 'result*List<'char>*'ctxt
    | Error of string
  \end{code}

  \subsubsection*{Monad Transformers}
  It may not be obvious, but the parser monad can be constructed from more primitive monads.
  The input stream and the context are represented as a state-monad, and the result is an either-monad.

  \begin{code}[language=Caml]
  type Parser<'char,'ctxt,'result> =
    State<List<'char>*'ctxt,
          Either<'result,'error>>
  \end{code}

  Combining monads by hand is tedious and non-trivial.
  Luckily there is a way to automate the process: monad transformers.\cite[Chapter~18]{Haskell}

  Monad transformers are \textit{type operators} (type-level functions) that takes a monad, and returns a combined monad.
  For example:
  if we pass the identity monad to the either monad transformer, we get an either monad.
  If we then pass the result monad to a state monad transformer, we get a result monad that can hold state.
  If we assign a list of input characters and a context to the state, we have constructed a generic parser monad.

  In a hypothetical language where we could write this, it woult look like this:
  
  \begin{code}[language=Caml]
    type ParserM = StateT (ResultT IdM)
  \end{code}
  
 Where as writing one by hand would require the type definitions, as well as the implementation \verb|return| and \verb|>>=|.

  \begin{code}[language=Caml]
  type Result<'char,'ctxt,'result> = 
    | Done  of 'result*List<'char>*'ctxt
    | Error of string
  \end{code}
  \begin{code}[language=Caml]
  type Parser<'char,'ctxt,'result> =
    List<'char> * 'ctxt
      -> Result<'char,'ctxt,'result>
  \end{code}
  \begin{code}[language=Caml]
  let return res =
    fun (chars,ctxt) ->
      Done(res,chars,ctxt)
  \end{code}
  \begin{code}[language=Caml]
  let (>>=) p k =
    fun (chars,ctxt) ->
      match p (chars,ctxt) with
      | Error(p) ->
          Error(p)
      | Done(res,chars',ctxt') ->
          k res (chars',ctxt')
  \end{code}

  \subsubsection*{Towards $F_\omega$}
  In order to have type operators, we need higher-order polymorphism.
  The formal name for a type system that supports higher-order polymorphism, System-F$_\omega$.\cite[Chapter~30]{Pierce02}
  Now that it is clear what is needed, we can show what language features we need to add.
  
  %Now that it is clear why monad-transformers are desirable, let's find out what type system we need to support them.
  %In order to have typeclasses like monads, we need subtyping($\lambda_{<:}$)\footcite[Chapter~15]{Pierce02}.
  %And in order to have type operators, we need higher-order polymorphism ($F_\omega$).\footcite[Chapter~30]{Pierce02}.
  %Combining these two, we get higher-order subtyping ($F^\omega_{<:}$)\footcite[Chapter~31]{Pierce02}.
  %It is this $F^\omega_{<:}$ that we will implement.
  
