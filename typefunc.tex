\subsection{TypeFunc Details}
TypeFuncs were designed to be consistent with the already existing Funcs.
This has a few implications:

\subsubsection*{TypeFuncs curry}
like Funcs, TypeFuncs curry.

Currying enables partially application.
For example, imagine we have a state monad.

\begin{code}
  TypeFunc "StateMonad" ⇒ ∗ ⇒ ∗ ⇒ ∗
\end{code}

\noindent Where the first argument represents the state type and the second the return type.
This means you can define a new state monad that already has its state-type specified by writing \verb|Monad(StateMonad int)|.

It also gives the compiler a uniform way to handle type and function application,
so this feature increases orthogonality while not increasing compiler complexity.

\subsubsection*{Dependent Types}
Dependent types are types that depend on terms, and are currently not supported by Meta-Casanova.
An example for this is the constant-length array, where the length of the array is embedded into the type at compile-time.

A future inclusion of dependent types is possible.
It would mean that the type namespace will be merged into the module namespace.
This won't break backward compatibility, since all identifiers in Meta-Casanova need to be unique.

\subsubsection*{Double arrows (⇒)}
We choose double arrows for TypeFuncs to clearly differentiate them from function application(→).
This distinction is to seperate run-time from compile-time.

This is unlike Haskell, it uses single arrows (→).
Haskell can do this because it does not have the explicit run-time/compile-time distinction Meta-Casanova has.

\subsubsection*{Typefuncs can have premises}
This means that TypeFuncs can behave just like rules.
This also gives us the ability to compute arbitrary expressions at compile-time.
The only limit is that values can't be passed at compile-time, as that would require dependent types.

\subsubsection*{TypeAliases}
\texttt{TypeAlias} declarations are the mechanism for declaring type-level two-way relationships between kinds.
This is analogous to the relationship \texttt{Data} declarations describe between types.
TypeAliases enable kind-level pattern-matching in TypeFuncs, essential for making them behave like Funcs.
